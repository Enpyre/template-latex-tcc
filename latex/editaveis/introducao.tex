\chapter[Introdução]{Introdução}

A programação tem se tornado cada vez mais essencial no mundo moderno, tanto no âmbito profissional quanto pessoal. No entanto, aprender a programar pode ser um desafio para muitas pessoas, especialmente para iniciantes que não possuem conhecimentos prévios sobre o assunto. Uma abordagem promissora para tornar o aprendizado de programação mais fácil e envolvente é a utilização de jogos, que permitem aos usuários aprender conceitos de programação em um contexto lúdico e visual.

A linguagem Python tem se tornado cada vez mais popular como uma linguagem de programação para iniciantes, devido à sua simplicidade e facilidade de uso. A combinação do Python com uma abordagem baseada em jogos pode ser particularmente eficaz para o ensino de programação para jovens estudantes e adultos iniciantes. A aprendizagem baseada em jogos pode aumentar a motivação e a retenção do conhecimento, tornando o processo de aprendizado mais envolvente e eficaz.

Nesse sentido, este trabalho apresenta a ferramenta Enpyre, uma plataforma de aprendizado de programação focada exclusivamente na linguagem Python, que oferece uma abordagem baseada em jogos para o ensino de programação. A ferramenta permite aos usuários criar jogos de forma fácil e intuitiva, utilizando uma interface gráfica e um editor de texto para escrever código Python. Além disso, a ferramenta utiliza a biblioteca PixiJS para fornecer uma interface gráfica para os jogos criados.

Ao permitir que os usuários criem jogos em Python, a ferramenta Enpyre oferece aos estudantes uma experiência de aprendizado prática e envolvente. A ferramenta é especialmente útil para educadores que desejam ensinar programação de forma eficaz e agradável, e para estudantes que desejam aprender programação de maneira interativa e lúdica. Este trabalho descreve em detalhes o funcionamento da ferramenta Enpyre e seus recursos, além de apresentar um estudo de caso que demonstra a eficácia da abordagem de ensino de programação baseada em jogos e a utilização do Enpyre.