\chapter[Introdução]{Introdução}
\addcontentsline{toc}{chapter}{Introdução}

A crescente demanda por profissionais de tecnologia fez com que muitos jovens e adultos despertassem o interesse pela área, e o Desenvolvimento de Software se destaca como sendo um dos mais almejados nichos de
tecnologia. No entanto, programação, um dos aspectos do Desenvolvimento de Software, é uma prática bastante difícil \cite{robins2003}.

São diversos os fatores relatados por professores e alunos que estão relacionados às dificuldades no processo de ensino e aprendizagem, seja de aspectos teóricos ou práticos das disciplinas \cite{marcolino2015}.%  Como consequência disso, os cursos que ensinam programação têm altas taxas de desistência e reprovação \cite{haddad2011, gomesmendes2007}.

Outra origem de dificuldades para o aprendizado são os requisitos técnicos necessários para se iniciar a programação com uma determinada tecnologia. Por exemplo, um estudante que esteja aprendendo a programar usando Java terá que primeiro entender e saber usar um editor de texto, um shell interativo e o Java Development Kit (JDK). Instalar o JDK exige, dependendo do sistema operacional em questão, a modificação das variáveis de ambiente PATH e CLASSPATH, e portanto o estudante deverá entender isso também \cite{truong2003}.


\section{Dificuldades Gerais}
	
O ensino de programação possui, intrinsecamente, várias dificuldades relacionadas à construção da lógica para resolução de problemas. Contudo, também existem dificuldades no entendimento das sintaxes de linguagens de programação e de determinados conceitos, tais como declaração de variáveis, estruturas de condições e repetições \cite{helminen2010}, aplicação dos conceitos durante a construção de um programa, desenvolvimento de habilidades de análise, compreensão de programas e motivação dos alunos \cite{robins2010}.

Outro obstáculo que pode ser observado nos estudos de programação está na escolha dos materiais usados para o ensino. Materiais de ensino tradicionais, como livros impressos ou slides, não são suficientes para transmitir a compreensão da programação, porque não são dinâmicos. Para a melhor compreensão deste estudo, é necessário que haja visualização, interação ao vivo e elementos dinâmicos para que o estudante consiga uma representação mental do problema a ser resolvido \cite{gomesmendes2007, cheah2020}.

O estudo de \citeauthor{bennedsencaspersen2005} (\citeyear{bennedsencaspersen2005}) e \citeauthor{zhangetall2013} (\citeyear{zhangetall2013}) corrobora ao afirmar que o uso de métodos tradicionais de ensino não parecem ser eficazes no ensino de programação de computadores.
	
Ao mesmo tempo, quando são usadas ferramentas dinâmicas para o ensino de programação, que auxiliem a visualização de um programa ou de modelos mentais, é preciso que sejam o mais simples possível. Ambientes de desenvolvimento que possuam muitas opções disponíveis criam dificuldades e atrasam o aprendizado, além de gerarem no estudante a percepção de que programação é mais difícil do que realmente é. Programadores novatos em geral se beneficiam de um ambiente de desenvolvimento mais simples \cite{MasonCooper2013, robins2019}.


\section{Dificuldades com a Linguagem de Programação}

O aprendizado também sofre interferência da própria linguagem de programação. A maior parte das linguagens de programação de uso comum são ferramentas desenvolvidas para uso profissional, e não para facilitarem o aprendizado. Como resultado, o estudante precisa dividir sua atenção entre aprender lógica de programação, algoritmos e resolução de problemas, e as complexidades relacionadas a sintaxe e semântica da linguagem escolhida \cite{gomesmendes2007}. No que diz respeito à pedagogia, os instrutores tendem a ensinar a sintaxe da linguagem no lugar de promover o método de resolução de problemas, e normalmente a linguagem escolhida é baseada na popularidade e não na adequação pedagógica \cite{gomesmendes2007}.

\citeauthor{brownwilson2018} (\citeyear{brownwilson2018}) e \citeauthor{savagepiwek2019} (\citeyear{savagepiwek2019}) afirmam que a escolha inadequada da linguagem de programação para fins educacionais prejudica a eficácia do aprendizado e aumenta ainda mais a dificuldade dos alunos no processo de compreensão do conteúdo. A programação exige um pensamento mais analítico e alto nível de abstração.. Desta forma, a aplicação de certos conceitos e algoritmos da linguagem escolhida deve ser fácil de lembrar, autoexplicativa e pouco complexa para facilitar o ciclo de aprendizagem durante o período de introdução \cite{gomesmendes2007, robins2019}.

\section{Soluções Existentes}
	
Mesmo diante dos desafios citados, existem muitas ferramentas disponíveis que podem auxiliar no processo de ensino e aprendizagem da programação. Contudo, pelo que se observa, muitos problemas permanecem sem solução \cite{bossegerosa2017, gomesmendes2007, robins2019, savagepiwek2019}. Aprender programação exige uma abordagem diferenciada, esforço contínuo e habilidades em várias camadas. Obter essas habilidades exige tentativa e erro e muita persistência \cite{jiauchenssu2009}.
	
A utilização de jogos educativos tornou-se comum para minimizar os problemas descritos. O termo \emph{jogos educativos} normalmente é associado a jogos de computador usados como ferramenta educacional, e que oferecem atividades interativas, envolventes e atraentes para os alunos \cite{gunterkennyvick2007}.  Esses jogos são populares e seu uso parece capturar mais a atenção do aluno \cite{barnesetall2007}, além de promover a criatividade, disposição para novas atividades, \cite{kellygibson2006}, motivação \cite{freitasjavris2006}, engajamento, e desempenho em sala de aula \cite{chaffinetall2009}.
	
Na literatura, vários ambientes são citados como meios para auxiliar o processo de ensino e aprendizagem de programação e algoritmo. A lista de exemplos inclui o JavaTool \cite{mota2008}, o Online Python Tutor \cite{guo2013}, o Pythy \cite{edwardsetall2013}, entre outros. Por mais que existam muitos trabalhos com essa temática, sempre é possível acrescentar novas funcionalidades e aperfeiçoamentos com o objetivo de proporcionar aos alunos experiências práticas significativas.

\section{Considerações Finais do Capítulo}
O presente capítulo abordou o cenário atual relacionado ao ensino e aprendizado de programação, salientando
sua importância e seus obstáculos. Tornaram-se evidentes os problemas associados a metodologias de ensino, linguagem de programação escolhida e materiais didáticos escolhidos. O capítulo expôs também algumas soluções já existentes para os problemas listados.
