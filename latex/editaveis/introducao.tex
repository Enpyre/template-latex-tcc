\chapter[Introdução]{Introdução}

A programação tem se tornado cada vez mais essencial no mundo moderno, tanto no âmbito profissional quanto pessoal. No entanto, aprender a programar pode ser um desafio para muitas pessoas, especialmente para iniciantes que não possuem conhecimentos prévios sobre o assunto. Uma abordagem promissora para tornar o aprendizado de programação mais fácil e envolvente é a utilização de jogos, que permitem aos usuários aprender conceitos de programação em um contexto lúdico e visual.

A linguagem Python tem se tornado cada vez mais popular como uma linguagem de programação para iniciantes, devido à sua simplicidade e facilidade de uso. A combinação do Python com uma abordagem baseada em jogos pode ser particularmente eficaz para o ensino de programação para jovens estudantes e adultos iniciantes. A aprendizagem baseada em jogos pode aumentar a motivação e a retenção do conhecimento, tornando o processo de aprendizado mais envolvente e eficaz.

O estudo de \citeauthor{PereiraEtAll} (\citeyear{PereiraEtAll}) levantou que "as ferramentas para programação estão entre os três maiores problemas relacionados ao ensino e aprendizagem do assunto". Além disso, "várias ferramentas e estratégias de ensino são implementadas na preparação de estudantes de programação, por meio de contos narrativos, jogos, simulações e técnicas de visualização".

Com isto posto, este trabalho tem por objetivo levantar as principais dificuldades envolvidas no aprendizado de programação, e propor uma ferramenta que as mitigue. A ferramenta proposta deve ser fácil, intuitiva, e possuir interface gráfica com elementos interativos relacionados a jogos.