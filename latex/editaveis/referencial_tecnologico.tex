\chapter[Referencial Tecnológico]{Referencial Tecnológico}
\addcontentsline{toc}{chapter}{Referencial Tecnológico}

O presente capítulo lista as tecnologias escolhidas ou envolvidas na criação da ferramenta.
da ferramenta Pyon. Para as ferramentas de escolha, são apresentadas suas definições, suas vantagens e 
suas limitações. Para as tecnologias associadas ao projeto por consequência, são apresentadas suas
definições.

\section{WebAssembly}

\subsection{Definição}

WebAssembly é um bytecode portátil de baixo nível para Web. É desenvolvido de modo a ser independente de hardware e plataforma. Com o WebAssembly, existe a possibilidade de se compilar código de diversas linguagens em código de baixo nível. Atualmente existe suporte para C, C++, Rust entre outros.

O WebAssembly foi proposto de forma colaborativa por engenheiros de quatro grandes distribuidores de navegadores: Google, Microsoft, Mozilla e Apple. A implementação atual já é capaz de compilar uma quantidade grande de programas, mas existem ainda várias propostas em estudo para estender as funcionalidades da linguagem e a integração com JavaScript.

\subsection{Vantagens}

O WebAssembly é seguro, rápido, independente de hardware e plataforma, além de ser portátil e compacto. Em comparação, outras tecnologias que tentam ou tentaram a execução de código nativo na Web, como o ActiveX da Microsoft, o Native Client da Google, o Java da Oracle e o Flash da Adobe, falham ao alcançar pelo menos uma destas características.

\subsection{Limitações}

Dentre as limitações presentes no WebAssembly, está sua incapacidade de acessar diretamente o DOM, então qualquer manipulação de DOM deve ser feita com JavaScript. Além disso, o WebAssembly funciona apenas com uma única thread, e não tem acesso a todas as funções do navegador.


\section{Pyodide}

\subsection{Definição}

O Pyodide é uma ferramenta que contém o interpretador do Python, em sua versão 3.9, sendo compilado para WebAssembly. Como resultado, é possível executar Python na Web, à partir de um navegador.

A ferramenta foi criada em 2018 por Michael Droettboom, desenvolvedor da Mozilla, inicialmente como parte do projeto Iodide. Este projeto propunha um ambiente de desenvolvimento baseado em navegadores, voltado para computação e programação científica e execução de notebooks do Jupyter. O projeto Iodide foi encerrado em Setembro de 2020, ainda podendo ser encontrado, mas não mais sendo mantido.

\subsection{Vantagens}

O Pyodide permite troca de informações entre os contextos JavaScript e Python. Por exemplo, é possível, usando JavaScript, manipular o DOM com informações obtidas do contexto do Python.

Para que a troca de informações entre contextos funcione, o Pyodide possui integrada uma conversão de tipos, que atua automaticamente durante o acesso de contexto do JavaScript para o Python ou vice-versa.

Além disso, é possível instalar dependências usando uma abstração do gerenciador de pacotes do Python, chamada “micropip”. É possível baixar dependências tanto do repositório oficial, quanto dependências hospedadas em outros ambientes, contanto que estejam no formato “wheel".

\subsection{Limitações}

Dentre as limitações do Pyodide, está o fato de que não é capaz de guardar cache dos pacotes baixados durante sua execução.

Além disso, não é possível trabalhar com multiprocessamento, threading ou sockets, devido a limitações do WebAssembly.

\section{PixiJS}

\subsection{Definição}

O PixiJS é um sistema de renderização para Web que usa WebGL (ou Canvas, opcionalmente) para a exibição de conteúdo visual 2D.

\subsection{Vantagens}

O PixiJS foi construído com o intuito de maximar a performance de execução no navegador do usuário.

Além da renderização de imagens, a biblioteca oferece a renderização de objetos primitivos como linhas, círculos, polígonos, bem como textos e sprites.

Diferente do Unity e do Flash, o PixiJS não requer a instalação de um plugin ou aplicação para funcionar, dependendo apenas do navegador do usuário.

\subsection{Limitações}

O PixiJS se limita a renderização de imagens. Não é possível usá-lo para renderizações 3D, armazenamento de dados e reprodução de áudio. O PixiJS não é um framework para jogos e não possui interface gráfica.


\subsection{Outras Tecnologias}

As escolhas do PixiJS e do Piodide para o projeto trazem interação com as seguintes tecnologias:

\subsubsection{Canvas API}
Trata-se de uma API que viabiliza o desenho de gráficos 2D à partir do JavaScript. É possível construir animações, gráficos para jogos, manipulação de fotos, processamento de vídeo, visualização de dados etc. Os elementos gerados pela Canvas API são renderizados no elemento "canvas" do HTML.

\subsubsection{WebGL API}

Semelhante à Canvas API, a WebGL API permite o desenho de gráficos em um elemento "canvas" do HTML.
Diferente da Canvas API, no entanto, os gráficos gerados pela WebGL API podem ser 2D e 3D, e contam
com aceleração de hardware.

\subsubsection{HTML Canvas, DOM, Event, Document}

\begin{itemize}

	\item \textbf{HTML Canvas}: elemento HTML usado para renderizar gráficos da Canvas API ou WebGL API.
	\item  \textbf{Event}: interface que representa um evento que ocorre no DOM. Um evento pode ser resultado
	de interação do usuário (clique de botão, por exemplo), gerado por APIs para notificar andamento
	de tarefas assíncronas (carregamento de página concluído, por exemplo) ou invocado programaticamente
	via JavaScript.
	\item  \textbf{DOM}: significa \textit{Document Object Model}, e representa o HTML carregado na
	forma de uma árvore de nós no navegador, onde cada nó representa uma parte do documento \textit{Web}.
	\item \textbf{Document}: trata-se de um objeto que representa a página \textit{Web} carregada no
	navegador, sendo a raíz do DOM.

\end{itemize}

\section{Considerações Finais do Capítulo}

O presente capítulo listou as tecnologias em uso pela Prova de Conceito do Pyon, expondo tanto as bibliotecas
escolhidas para comporem sua estrutura principal (PixiJS, Pyodie), bem como as tecnologias usadas por elas.