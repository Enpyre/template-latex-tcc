\chapter*[Resultados]{Resultados}
\addcontentsline{toc}{chapter}{Resultados}

O presente capítulo exibe os resultados obtidos após a concepção da Prova de Conceito.

\begin{figure}[!ht]
    \centering
    \caption{Move Ball Game}
    \includegraphics[keepaspectratio=true,scale=0.6]{figuras/code_example.eps}
    \legend{Fonte: Autores}
    \label{fig:code}
\end{figure}

\section{Aparencia}

A Figura \ref{fig:code} exibe um navegador aberto e dois elementos principais: um canvas, acima, contendo a saída do programa; e um editor de texto, abaixo, contendo o código escrito pelo usuário. O resultado do programa permite que o usuário interaja com o círculo desenhado, usando as teclas direcionais para alterar a direção em que a bola se move. 

\section{Nome da Ferramenta: Pyon}

O nome Pyon teve a origem descrita a seguir. Primeiro, pensou-se na ferramenta como um "motor" 
(\textit{engine}, em inglês) de jogos escrito em Python. Este conceito foi alterado no princípio do trabalho, mas por sua razão adotou-se o nome provisório \textbf{Pyengine}.

Posteriormente, quando se consolidou o uso da biblioteca Pyodide, o nome provisório foi alterado para
\textbf{Pyongine}, uma mesca do nome anterior com o nome da biblioteca.

Por fim, buscando-se simplificar o nome que se usava até então, reduziu-se Pyongine para \textbf{Pyon},
acreditando-se que sua pronúncia fosse mais agradável.

\section{Estrutura Visual}

A aplicação é constituída por duas partes principais: um canvas, onde elementos podem ser desenhados, e um editor de texto integrado, com realce de sintaxe, onde o código Python pode ser escrito e executado.

\section{Resultados}

A prova de conceito comprovou a viabilidade do projeto, que mistura elementos e contextos do JavaScript e Python graças a dois proxies existentes no Pyodide.

Foi possível desenhar círculos coloridos no Canvas à partir de código Python escrito e executado à partir do editor de texto integrado. A execução segue a ordem a seguir.

\begin{enumerate}
    \item O código Python é extraído do editor de texto via JavaScript.
    \item O código Python então é interpretado pelo Pyodide, e executado em seguida.
    \item A execução do código Python inclui o download, via gerenciador de pacotes do Python (pip), do módulo PyonPy.
    \item As funções constantes neste módulo são capazes de acessar, graças ao Pyodide, a interface
    document do JavaScript. 
    \item Através desta interface, as funções do PixiJS podem ser executadas. E finalmente, pode-se desenhar gráficos 2D no Canvas do DOM. As funções da parte Python do módulo Pyon expõe estas funcionalidades de uma maneira conveniente.
\end{enumerate}

\section{Considerações Finais do Capítulo}
Este capítulo confirmou a viabilidade da Prova de Conceito do Pyon, e relatou a ordem de execução 
de um comando Python realizado no editor de texto integrado.