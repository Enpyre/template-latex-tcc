\chapter*[Considerações Finais]{Considerações Finais}
\addcontentsline{toc}{chapter}{Considerações Finais}

\section{Status Atual}

A viabilidade do Pyon representa um sucesso na primeira etapa do presente trabalho. 

Acreditamos que o Pyon, uma vez que esteja plenamente desenvolvido, seja mais eficiente para o ensino de programação do que métodos tradicionais e outras ferramentas de propósito similar.
O Pyon utiliza uma linguagem de programação de sintaxe simples quando comparada com C, C++, Java, dentre outras, e é executado à partir de um navegador, o que diminui complicações advindas da configuração de ambiente e instalação de dependências por parte do aluno. Além disso, apresenta elementos visuais dinâmicos, o que facilita o entendimento de determinados conceitos básicos, como laços de repetição e estruturas condicionais.

\section{Próximas Etapas}

As próximas etapas do projeto consistem na implementação da ferramenta Pyon, à partir da Prova de Conceito realizada. Se possível, espera-se acompanhar seu uso em um ambiente de aprendizado real.

\subsection{Oportunidades de Acompanhamento}

A Pubnic é uma startup sediada em Brasília, DF, cujo serviço consiste no ensino de
programação para alunos com pouca ou nenhuma experiência, e sua subsequente entrada no
mercado de trabalho, em um modelo que se assemelha a uma Residência para desenvolvedores.
Uma vez desenvolvido, o Pyon será aplicado para um grupo de alunos da Pubnic, e métricas de
desempenho serão coletadas e comparadas com outro grupo onde a ferramenta não tenha sido
disponibilizada. Depoimentos dos alunos também serão coletados.