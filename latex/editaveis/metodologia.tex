\chapter[Metodologia]{Metodologia}

\section{Metodologia de Desenvolvimento de Software}

Um processo ou metodologia de desenvolvimento de software é uma coleção de atividades e resultados relacionados que auxiliam na criação de software. Por exemplo, análise e codificação de requisitos são duas das muitas atividades associadas. \cite{Soares2004}

Há vários tipos de metodologia de desenvolvimento de software como o Scrum, PMBOK, Kanban, Ágil, Ciclo de Vida Único, Modelo em cascata, entre outras. Cada uma delas possui suas próprias características e abordagens para gerenciamento de projetos, mas todas elas visam ajudar as equipes a entregar projetos de software de qualidade dentro do prazo, orçamento e com as funcionalidades esperadas. Algumas metodologias são mais estruturadas e baseadas em fases, enquanto outras são mais flexíveis e baseadas em fluxo de trabalho, o que torna importante escolher a metodologia mais adequada para o projeto específico. \cite{SilvaAnd2013}

\subsection{Metodologia Ágil}

As metodologias ágeis são um conjunto de abordagens para gerenciamento de projetos que se concentram em fornecer flexibilidade, adaptabilidade e colaboração entre equipes. Elas foram desenvolvidas originalmente para o desenvolvimento de software, mas agora são amplamente utilizadas em muitos outros campos. \cite{Libardi2010}

\subsubsection{Scrum}

Scrum é uma metodologia ágil específica que foi projetada para ajudar equipes a gerenciar projetos de desenvolvimento de software complexos e altamente incertos. Ele se concentra em entregar valor rapidamente e incrementos regulares, enquanto a equipe colabora e se adapta a mudanças no projeto. \cite{STOPA2019}

\subsubsection{Kanban}

Kanban é uma metodologia que se concentra em tornar visível o fluxo de trabalho, limitando o número de tarefas em andamento e permitindo que a equipe se adapte às mudanças de forma flexível. Ele é amplamente utilizado em conjunto com outras metodologias ágeis, como Scrum, para ajudar a equipe a aumentar a eficiência e a entrega de valor. \cite{Lage2008}

\subsection{Metodologia de Desenvolvimento Utilizada}

Kanban e Scrum são ambas metodologias ágeis, mas possuem abordagens diferentes para o gerenciamento de projetos. Enquanto o Scrum é uma metodologia mais estruturada e baseada em sprints, Kanban é uma metodologia mais flexível e baseada no fluxo de trabalho. Ambos têm seus próprios benefícios e podem ser usados de forma complementar para ajudar a equipe a alcançar seus objetivos de projeto. \cite{Kniberg2010}

Sendo assim, o projeto foi desenvolvido utilizando ambas as metodologias, Kanban e Scrum, para que fosse possível atingir os objetivos propostos. O Kanban foi utilizado para gerenciar o fluxo de trabalho, enquanto o Scrum foi utilizado para gerenciar as sprints.

\section{Ferramentas Utilizadas}

\subsection{Gerenciamento do Projeto}

\subsubsection{Trello}

Trello é uma plataforma que auxilia equipes a organizarem e priorizarem projetos. Utiliza quadros de tarefas, onde cada tarefa é representada por um cartão, que pode ser movido e organizado em listas para indicar o progresso e o estado do projeto. Ele é uma ferramenta útil para gerenciar tarefas do projeto, estabelecer prazos e definir prioridades, além de acompanhar o andamento de cada Sprint.

\subsubsection{Slack}

Slack é uma ferramenta de comunicação de equipe que permite que as equipes criem canais de bate-papo para discutir projetos, compartilhar arquivos e se comunicar de forma rápida e eficiente.

No contexto do desenvolvimento de software, Slack pode ajudar a equipe a se comunicar de forma mais eficiente e colaborativa, permitindo que desenvolvedores, gerentes de projetos e outros membros da equipe discutam e compartilhem informações de forma rápida e fácil. Ele pode ser usado para discutir problemas técnicos, atribuir tarefas, acompanhar o progresso do projeto e compartilhar arquivos, tudo em um único lugar. Além disso, ele pode ser integrado com outras ferramentas, como o GitHub, o Google Drive e o Trello, para ajudar a equipe a automatizar tarefas e trabalhar de forma mais eficiente.

\subsubsection{Google Drive}

Google Drive é um serviço de armazenamento e compartilhamento de arquivos na nuvem fornecido pelo Google. Ele permite que os usuários armazenem arquivos, como documentos, fotos e vídeos, e compartilhem esses arquivos com outras pessoas. Ele também oferece recursos de colaboração, como edição em tempo real de documentos, comentários e histórico de versões.

\subsection{Desenvolvimento}

\subsubsection{Figma}

Figma é uma ferramenta de design colaborativo que permite que equipes criem e compartilhem arquivos de design. Ele permite que os usuários criem protótipos de aplicativos e sites, além de permitir que eles compartilhem esses protótipos com outras pessoas para que possam colaborar e fazer comentários. Ele também permite que os usuários criem e compartilhem arquivos de design, como imagens, ícones e paletas de cores, que podem ser usados por outros membros da equipe.

\subsubsection{GitHub e GitHub Actions}

GitHub é uma plataforma de desenvolvimento de software que permite que os desenvolvedores armazenem, rastreiem e colaborem em projetos de código-fonte. Ele é baseado no sistema de controle de versão Git, que permite que os desenvolvedores façam alterações no código e versionem essas alterações, facilitando a colaboração e o controle de versão.

GitHub Actions é uma ferramenta de automação de fluxo de trabalho do GitHub. Ele permite que os desenvolvedores criem scripts de trabalho automatizados, chamados "Ações", que podem ser disparados por eventos, como o envio de código para o repositório ou a criação de uma nova issue. Isso permite automatizar tarefas comuns, como compilação, teste e implantação, e integrar com outras ferramentas, como ferramentas de integração contínua e de monitoramento de desempenho.

\subsubsection{Docker}

Docker é uma plataforma de virtualização de aplicativos que permite que os desenvolvedores embalem e distribuam facilmente aplicativos em contêineres. Um contêiner é uma forma de isolamento de sistemas operacionais que permite que um aplicativo seja embalado com todas as suas dependências, bibliotecas e configurações, de forma que possa ser executado de forma consistente em diferentes ambientes.

Docker permite que os desenvolvedores criem e gerenciem contêineres, e que esses contêineres sejam implantados em diferentes ambientes, incluindo computadores locais, nuvens públicas e privadas. Isso permite que os desenvolvedores criem aplicativos de forma mais eficiente e confiável, e que esses aplicativos possam ser executados de forma consistente em diferentes ambientes. Além disso, ele também permite a colaboração entre equipes, e aumenta a segurança, escalabilidade e gerenciamento do aplicativo.

\subsubsection{Visual Studio Code}

VSCode (Visual Studio Code) é um editor de código-fonte desenvolvido pela Microsoft. Ele é uma ferramenta gratuita e de código aberto, que suporta diversas linguagens de programação, incluindo JavaScript, Python, C++, Java e muitas outras. Ele oferece recursos avançados de edição, como sugestão de código, depuração e gerenciamento de versão. Além disso, ele também possui uma ampla variedade de extensões e plugins desenvolvidos pela comunidade, que podem ser usadas para adicionar recursos adicionais, como integração com ferramentas de teste, análise de código e integração com outras ferramentas.

\section{Definição de Escopo e Requisitos}

A definição de escopo e requisitos é uma etapa crítica no processo de gerenciamento de projetos, pois estabelece as expectativas e restrições do projeto, bem como os objetivos e entregáveis esperados.

O escopo do projeto é o conjunto de tarefas, atividades e entregáveis que precisam ser realizados para completar o projeto de acordo com as necessidades do cliente. Ele também inclui os limites e restrições do projeto, como tempo, orçamento e recursos disponíveis. A definição do escopo do projeto é importante para garantir que todos os envolvidos tenham uma compreensão clara do que é e não é incluído no projeto. \cite{Xavier2009}

Os requisitos são as necessidades e expectativas do cliente e do usuário final que devem ser atendidas pelo projeto. Eles são usados para guiar o desenvolvimento do projeto e incluem tanto requisitos funcionais (o que o sistema deve fazer) quanto requisitos não funcionais (como desempenho, segurança e usabilidade). A definição de requisitos é importante para garantir que o projeto entregue o valor esperado para o cliente e usuário final. \cite{Machado2018}

Em resumo, a definição de escopo e requisitos é importante para garantir que o projeto seja entregue dentro do prazo, orçamento e com as funcionalidades esperadas, e que ele atenda as necessidades do cliente e usuário final.

\subsection{Definição de Escopo}

Para a definição de escopo do projeto, foi realizada uma reunião com o professor orientador, onde foram discutidos os requisitos do projeto, e o escopo do projeto foi definido. O escopo do projeto inclui a criação de uma plataforma web para o ensino de Python com recursos interativos e lúdicos. A plataforma deve ser capaz de oferecer um ambiente de aprendizado para os alunos, com a possibilidade de criar e compartilhar projetos, e também deve ser capaz de oferecer um ambiente de ensino para os professores, com a possibilidade de criar e compartilhar aulas.

\subsection{Definição de Requisitos}

Os requisitos do projeto foram definidos com base nas necessidades do cliente e usuário final, e também com base nas restrições do projeto. Os requisitos do projeto incluem:

\begin{itemize}
    \item Criação de projetos: a plataforma deve permitir que os alunos criem e compartilhem projetos em Python.
    \item Criação de aulas: a plataforma deve permitir que os professores criem e compartilhem aulas em Python.
    \item Ambiente de aprendizado interativo: a plataforma deve oferecer um ambiente de aprendizado interativo e lúdico para os alunos, com recursos como quizzes e desafios para ajudar a fixar o conteúdo.
    \item Gerenciamento de conteúdo: a plataforma deve permitir que os professores gerencie e atualize as aulas e projetos criados.
    \item Comunicação entre alunos e professores: a plataforma deve permitir a comunicação entre alunos e professores, como fóruns de discussão, chats e comentários.
    \item Análise de desempenho: a plataforma deve fornecer relatórios e gráficos para acompanhar o progresso dos alunos e identificar áreas de melhoria, e também deve permitir que os professores acompanhem o desempenho de seus alunos.
    \item Responsividade e acessibilidade: a plataforma deve ser projetada para funcionar em diferentes e navegadores, e deve seguir as normas de acessibilidade para garantir que seja acessível para todos os usuários.
    \item Segurança: a plataforma deve ser segura, com criptografia de dados e autenticação de usuários.
\end{itemize}

\subsubsection{Priorização dos Requisitos}

Os requisitos do projeto foram priorizados com base no impacto ao usuário final através da regra de Pareto, que afirma que 80\% do valor de um produto ou serviço é gerado por 20\% dos seus recursos. \cite{Xavier2009}

Esse projeto foi desenvolvido com base nos 20\% dos requisitos que geram 80\% do valor para o usuário final, ficando os outros 80\% dos requisitos como débitos técnicos. Os requisitos prioritários são:

\begin{itemize}
    \item Criação de projetos: a plataforma deve permitir que os alunos criem e compartilhem projetos em Python.
    \item Ambiente de aprendizado interativo: a plataforma deve oferecer um ambiente de aprendizado interativo e lúdico para os alunos, com recursos como quizzes e desafios para ajudar a fixar o conteúdo.
    \item Segurança: a plataforma deve ser segura, com criptografia de dados e autenticação de usuários.
\end{itemize}

\subsection{Prova de Conceito}

Para garantir a viabilidade da solução imaginada, criou-se uma aplicação web simplificada que faz uso do Pyodide e do PixiJS. Seu objetivo foi o de oferecer um editor de texto integrado, com o qual o usuário pudesse executar código python e interagir com a aplicação resultante. A aplicação permitiu o desenho de figuras geométricas de tamanhos e cores variadas dentro de um canvas.

O design da aplicação não foi uma preocupação, uma vez que seu principal objetivo era o de comprovar a viabilidade.

\subsubsection{Referencial Tecnológico}

No apêndice \ref{apendice:referencial_tecnologico_poc} encontra-se o referencial tecnológico da aplicação de Prova de Conceito.

\subsubsection{Arquitetura da Aplicação}

No apêndice \ref{apendice:arquitetura_poc} encontra-se a arquitetura da aplicação de Prova de Conceito.

\subsubsection{Resultados}

No apêndice \ref{apendice:resultados_poc} encontra-se os resultados da aplicação de Prova de Conceito.

