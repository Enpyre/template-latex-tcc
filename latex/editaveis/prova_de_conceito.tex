\chapter[Prova de Conceito]{Prova de Conceito}

\section{Visão Preliminar}

Com base nas dificuldades levantadas durante a concepção deste projeto, percebe-se uma carência de ferramentas que promovam o ensino de programação e que visem a diminuição de obstáculos comuns ao aprendizado tradicional. Tendo isto em vista, propõe-se a elaboração de um software que apresente elementos dinâmicos, interface simples, que seja de fácil distribuição e que ofereça interação com uma linguagem de programação que propicie o aprendizado.

\section{A Distribuição do Software}

Com o propósito de reduzir as dificuldades associadas com o preparo do ambiente de desenvolvimento, é desejável que a ferramenta proposta possua o mínimo de dependências possível, e que seja compatível com qualquer sistema operacional. Uma forma de se atingir este objetivo é usar uma solução que seja executada à partir do navegador.

A distribuição padrão do Python, conhecida como CPython, é implementada em C, e não roda nativamente no navegador. Existem poucas soluções que viabilizem isso, sendo o WebAssembly a mais completa. Trata-se de um padrão que define um formato de código binário de baixo nível capaz de ser executado à partir de um navegador. É possível compilar o interpretador de Python para WebAssembly, tornando possível sua execução em ambiente Web.

No entanto, existem vários desafios de integração entre o código do interpretador e o ambiente do navegador, e por isso não basta uma simples recompilação. O projeto Pyodide não só prepara a compilação do interpretador do Python, como também tem uma série de funções para integrar melhor os dois ambientes e promover contexto compartilhado entre Python e JavaScript.

Buscou-se então uma biblioteca JavaScript capaz de desenhar em um canvas com HTML5. Dentre as opções, escolheu-se o PixiJS. O objetivo passou a ser invocar o PixiJS à partir de código Python, de modo que o estudante pudesse desenhar formas geométricas, linhas e vértices com linhas de código simples.

\section{Prova de Conceito}

Para garantir a viabilidade da solução imaginada, criou-se uma aplicação web simplificada que faz uso do Pyodide e do PixiJS. Seu objetivo era o de oferecer um editor de texto integrado, com o qual o usuário pudesse executar código python e interagir com a aplicação resultante. A aplicação permitiu o desenho de figuras geométricas de tamanhos e cores variadas dentro de um canvas.

O design da aplicação não foi uma preocupação, uma vez que seu principal objetivo era o de comprovar a viabilidade.

\subsection{Referencial Tecnológico}

No apêndice \ref{apendice:referencial_tecnologico_poc} encontra-se o referencial tecnológico da aplicação de Prova de Conceito.

\subsection{Arquitetura da Aplicação}

No apêndice \ref{apendice:arquitetura_poc} encontra-se a arquitetura da aplicação de Prova de Conceito.

\subsection{Resultados}

No apêndice \ref{apendice:resultados_poc} encontra-se os resultados da aplicação de Prova de Conceito.

