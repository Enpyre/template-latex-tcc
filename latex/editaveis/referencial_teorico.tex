\chapter[Referencial Teórico]{Referencial Teórico}
\addcontentsline{toc}{chapter}{Referencial Teórico}

\section{Programação}

Lorem Ipsum

\section{Cursos de Programação}

Lorem Ipsum

\section{Dificuldades Durante o Aprendizado}

São diversos os fatores relatados por professores e alunos que estão relacionados às dificuldades no processo de ensino e aprendizagem, seja de aspectos teóricos ou práticos das disciplinas \cite{marcolino2015}.%  Como consequência disso, os cursos que ensinam programação têm altas taxas de desistência e reprovação \cite{haddad2011, gomesmendes2007}.

\subsection{Linguagem e Lógica}

Quando estudando programação, os estudantes podem encontrar dificuldades relacionadas à construção da lógica para a resolução de problemas, no entendimento das sintaxes de linguagens de programação, no desenvolvimento de habilidades de análise e compreensão de programas \cite{robins2010} e em determinados conceitos como declaração de variáveis e estruturas de condições e repetições \cite{helminen2010}.

Outra origem de dificuldades para o aprendizado são os requisitos técnicos necessários para se iniciar a programação com uma determinada tecnologia. Por exemplo, um estudante que esteja aprendendo a programar usando Java terá que primeiro entender e saber usar um editor de texto, um shell interativo e o Java Development Kit (JDK). Instalar o JDK exige, dependendo do sistema operacional em questão, a modificação das variáveis de ambiente PATH e CLASSPATH, e portanto o estudante deverá entender isso também \cite{truong2003}.

\subsection{Materiais}

Outro obstáculo que pode ser observado nos estudos de programação está na escolha dos materiais usados para o ensino. Materiais de ensino tradicionais, como livros impressos ou slides, não são suficientes para transmitir a compreensão da programação, porque não são dinâmicos. Para a melhor compreensão deste estudo, é necessário que haja visualização, interação ao vivo e elementos dinâmicos para que o estudante consiga uma representação mental do problema a ser resolvido \cite{gomesmendes2007, cheah2020}.

O estudo de \citeauthor{bennedsencaspersen2005} (\citeyear{bennedsencaspersen2005}) e \citeauthor{zhangetall2013} (\citeyear{zhangetall2013}) corrobora ao afirmar que o uso de métodos tradicionais de ensino não parecem ser eficazes no ensino de programação de computadores.

Por outro lado, quando são usadas ferramentas dinâmicas para o ensino de programação que auxiliem a visualização de um programa ou de modelos mentais, é preciso que sejam o mais simples possível. Ambientes de desenvolvimento que possuam muitas opções disponíveis criam dificuldades e atrasam o aprendizado, além de gerarem no estudante a percepção de que programação é mais difícil do que realmente é. Programadores novatos em geral se beneficiam de um ambiente de desenvolvimento mais simples \cite{MasonCooper2013, robins2019}.

\subsection{Soluções Existentes}

% Existem muitas ferramentas disponíveis que podem auxiliar no processo de ensino e aprendizagem da programação. Contudo, pelo que se observa, muitos problemas permanecem sem solução \cite{bossegerosa2017, gomesmendes2007, robins2019, savagepiwek2019}. Aprender programação exige uma abordagem diferenciada, esforço contínuo e habilidades em várias camadas. Obter essas habilidades exige tentativa e erro e persistência \cite{jiauchenssu2009}.

A utilização de jogos educativos tornou-se comum para minimizar os problemas descritos. O termo \emph{jogos educativos} normalmente é associado a jogos de computador usados como ferramenta educacional, e que oferecem atividades interativas, envolventes e atraentes para os alunos \cite{gunterkennyvick2007}.  Esses jogos são populares e seu uso parece capturar mais a atenção do aluno \cite{barnesetall2007}, além de promover a criatividade, disposição para novas atividades, \cite{kellygibson2006}, motivação \cite{freitasjavris2006}, engajamento, e desempenho em sala de aula \cite{chaffinetall2009}.

Vários ambientes são citados na literatura como meios para auxiliar o processo de ensino e aprendizagem de programação e algoritmo. A lista de exemplos inclui o JavaTool \cite{mota2008}, o Online Python Tutor \cite{guo2013}, o Pythy \cite{edwardsetall2013}, entre outros. % Por mais que existam muitos trabalhos com essa temática, sempre é possível acrescentar novas funcionalidades e aperfeiçoamentos com o objetivo de proporcionar aos alunos experiências práticas significativas.