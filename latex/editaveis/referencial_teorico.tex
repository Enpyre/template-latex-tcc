\chapter[Referencial Teórico]{Referencial Teórico}

\section{Programação}

Programação é uma das habilidades fundamentais do mundo moderno, com aplicações em diversas áreas, incluindo ciência, tecnologia e negócios. Em termos simples, a programação envolve a criação de algoritmos que orientam um computador a executar uma tarefa específica.

De acordo com \cite{VanRoy2004}, programação pode ser definida como "o processo de especificar um algoritmo computacional para resolver um problema". Um algoritmo é uma sequência lógica de instruções que descrevem como um problema deve ser resolvido. Essas instruções são escritas em uma linguagem de programação que é compreensível tanto para humanos quanto para computadores.

A programação é uma habilidade valiosa para o mundo moderno, pois permite que as pessoas criem soluções personalizadas para problemas específicos. Ela é utilizada em uma ampla variedade de aplicações, incluindo desenvolvimento de software, análise de dados, inteligência artificial, jogos, aplicações móveis e muito mais. Além disso, a programação ajuda a desenvolver habilidades importantes, como resolução de problemas, pensamento lógico e criatividade.

\section{Linguagens de Programação}

Uma linguagem de programação é uma linguagem formal que permite a um programador expressar instruções precisas para um computador ou outro dispositivo. As linguagens de programação são usadas para desenvolver software, aplicativos e sistemas, e são criadas com base em uma série de regras sintáticas e semânticas que permitem que os programadores criem códigos que possam ser entendidos e executados pelo computador. \citeauthor{MichaelTGoodrich} (\citeyear{MichaelTGoodrich}) define linguagem de programação como "uma notação para descrever algoritmos e estruturas de dados", dizendo também que são "uma coleção de símbolos, palavras-chave e regras sintáticas que definem como esses símbolos podem ser combinados para criar programas de computador".

Uma das barreiras no aprendizado de programação é a sintaxe da linguagem escolhida. \citeauthor{stefik2013} (\citeyear{stefik2013}) concluiu em sua investigação empírica sobre sintaxes de linguagem de programação que linguagens com sintaxe derivada ou semelhante a C, como C++ e Java, podem apresentar aspectos que dificultam o aprendizado de novos programadores, como a necessidade de inicializar e delcarar o tipo de variáveis, necessidade de se usar chaves, ponto e vírgula e expressões verbosas \cite{mannila2006}.

Python é uma linguagem de programação que abandonou as familiaridades de sintaxe com C \cite{stefik2013}, optando por usar a indentação ao invés de chaves para delimitar blocos de código. Python também foi alvo de um estudo por \citeauthor{jayal2015} (\citeyear{jayal2015}) acerca de seu uso no ensino de programação introdutória, onde se concluiu que houveram melhores resultados no aprendizado quando os conceitos introdutórios (fluxos de controle e uso de bibliotecas, por exemplo) foram ensinados com Python, em comparação ao Java.

O Python adotou a indentação como elemento de sintaxe, o que torna os programas mais legíveis e fáceis de compreender. Além disso, é multiplataforma, possuindo suporte para sistemas como Windows, Linux, MacOS, etc; e possui módulos prontos em sua biblioteca principal para realização de várias tarefas comuns de programação \cite{moraispires2002}.

De acordo com \citeauthor{fangohr2004} (\citeyear{fangohr2004}), esta linguagem é indicada para disciplinas introdutórias e seu ambiente auxilia no aprendizado prático, seja na modalidade à distância ou em laboratórios.

\section{Métodos de Ensino de Programação}

A revisão sistemática da literatura realizada por \cite{Silva2014} e por \cite{Silva2018} identificou diversas abordagens e estratégias utilizadas para o ensino de programação.

\subsection{Escolha da linguagem de programação}

A maioria dos estudos analisados por \cite{Silva2014} utilizou linguagens de programação específicas para o ensino de programação, como Scratch, Alice e Python. Essas linguagens são escolhidas por serem simples e intuitivas, permitindo que alunos iniciantes possam aprender a programar de maneira mais fácil e rápida. Além disso, elas são visualmente atrativas, o que torna o processo de aprendizado mais lúdico e divertido.

\subsection{Abordagem Instrucionista}

A abordagem instrucionista, baseada em aulas expositivas e resolução de exercícios, ainda é a mais utilizada pelos professores brasileiros. De acordo com \cite{Silva2018}, essa abordagem é criticada por não incentivar a criatividade e a autonomia dos alunos, limitando-se apenas à reprodução de exemplos dados em sala de aula. No entanto, ela ainda é utilizada por ser considerada uma forma simples e eficaz de transmitir o conhecimento.

\subsection{Abordagem Construtivista}

A abordagem construtivista, na qual os alunos são encorajados a resolver problemas reais e a trabalhar em projetos de programação que envolvam outras áreas do conhecimento, é vista como uma alternativa à abordagem tradicional. Segundo \cite{Silva2018}, essa abordagem permite que os alunos desenvolvam habilidades de pensamento crítico e de trabalho em equipe, além de estimular a criatividade e a autonomia.

\subsection{Outros Métodos}

Além das abordagens mencionadas anteriormente, \cite{Silva2018} identificaram outros métodos de ensino de programação utilizados no Brasil, como a utilização de jogos educativos, a programação por blocos e a programação por agentes. Esses métodos são considerados alternativas às abordagens mais tradicionais e podem ser utilizados para tornar o processo de aprendizado de programação mais atrativo e dinâmico.


\section{Dificuldades no Aprendizado}

O ensino e aprendizado de programação apresentam diversos desafios e dificuldades. Dentre eles, destaca-se a abstração, que pode ser considerada como uma das principais barreiras para a aprendizagem de programação \cite{Silva2014}. De acordo com os autores, a habilidade de abstração é fundamental para o desenvolvimento de programas eficientes e de alta qualidade. Entretanto, a capacidade de abstrair conceitos complexos não é inata e pode ser um desafio para muitos estudantes.

Outra dificuldade importante no ensino e aprendizado de programação é a falta de motivação dos estudantes \cite{Silva2014}. Muitos alunos não conseguem visualizar a utilidade da programação em suas vidas e, portanto, não se sentem motivados a aprender. Segundo os autores, é importante que os professores expliquem a relevância e a aplicabilidade da programação para ajudar os alunos a se engajarem com o conteúdo.

Além disso, a falta de experiência prévia em lógica de programação pode dificultar a compreensão dos conceitos \cite{Silva2014}. Para muitos estudantes, a lógica por trás da programação pode não ser óbvia e pode exigir uma compreensão abstrata. Nesse sentido, é importante que os professores trabalhem com exemplos e exercícios que possam ajudar os alunos a compreender os conceitos fundamentais.

A complexidade das linguagens de programação, especialmente para iniciantes, também pode representar um desafio significativo \cite{Silva2014}. As linguagens de programação podem incluir uma ampla variedade de recursos e comandos, e a compreensão de sua sintaxe pode ser difícil para quem está começando. Para contornar essa dificuldade, é recomendado que os professores utilizem linguagens de programação mais simples e adequadas para iniciantes.

\section{Ensino com Jogos}

De acordo com {Moreira2018}, os jogos digitais podem ser uma excelente forma de engajar os alunos no aprendizado de programação. Diferentemente dos métodos tradicionais de ensino, que muitas vezes são enfadonhos e pouco interativos, os jogos digitais podem tornar o processo de aprendizado mais dinâmico e divertido, além de estimular o pensamento crítico e o raciocínio lógico dos alunos.

Ao mesmo tempo, a construção de jogos é uma abordagem interessante para o ensino de programação, uma vez que permite que os alunos apliquem seus conhecimentos na prática, criando seus próprios jogos. De acordo com o artigo "O Uso de Jogos Digitais para o Ensino de Programação" \cite{Moreira2018}, a construção de jogos é uma forma atraente e motivadora para ensinar programação aos alunos, pois permite que eles criem algo concreto e apliquem seus conhecimentos de programação na prática.

Para ensinar programação através da construção de jogos, é importante que os alunos tenham um conhecimento prévio de conceitos fundamentais de programação, como variáveis, loops e condicionais, para que possam aplicá-los na criação de seus jogos. O artigo "Jogos Digitais como Ferramenta de Apoio ao Ensino de Programação" \cite{Pereira2016} destaca a importância do uso de jogos na fase inicial de aprendizado da programação, uma vez que o ato de jogar pode ajudar os alunos a compreender melhor a lógica de programação e a entender como os conceitos são aplicados na prática.

Já o ato de jogar pode ser utilizado como uma ferramenta complementar ao ensino de programação. Conforme destacado por \cite{Pereira2016}, jogar jogos pode ajudar os alunos a compreender melhor a lógica de programação e a entender como os conceitos são aplicados na prática. Além disso, jogar jogos pode ser uma forma interessante de estimular o interesse dos alunos pela programação e de motivá-los a aprender mais sobre o tema.

Para utilizar o ato de jogar como ferramenta pedagógica, é importante escolher jogos adequados e que possam ser utilizados para ensinar conceitos de programação. Por exemplo, jogos de raciocínio lógico e de resolução de problemas podem ser utilizados para ensinar conceitos como condicionais, loops e variáveis, enquanto jogos de plataforma podem ser utilizados para ensinar conceitos de física e mecânica de movimento.

\section{Plataformas existentes}

Um exemplo de jogo digital que pode ser utilizado para o ensino de programação é o Code.org, uma plataforma que oferece uma série de jogos e atividades voltados para o aprendizado da lógica de programação. Os jogos do Code.org são projetados para incentivar os alunos a resolver problemas de programação de uma forma mais criativa e interativa, ajudando-os a desenvolver habilidades de programação mais avançadas ao longo do tempo.

Outra plataforma muito utilizada no ensino de programação é o Scratch, um software gratuito desenvolvido pelo MIT Media Lab. O Scratch é um ambiente de programação visual que permite aos alunos criar seus próprios jogos, histórias e animações. Os alunos podem arrastar e soltar blocos de programação para criar suas próprias sequências de comandos, sem precisar digitar códigos complexos. Essa abordagem torna o processo de aprendizado mais acessível para alunos iniciantes e ajuda a estimular a criatividade e o pensamento lógico.

O Blockly é outro exemplo de plataforma de programação visual que pode ser utilizada para o ensino de programação. Ele permite que os alunos criem seus próprios programas, usando blocos de código que podem ser facilmente arrastados e soltos para criar uma sequência de comandos. O Blockly é um ambiente de programação flexível, que pode ser usado para criar jogos, animações, aplicativos e muito mais.


Além desses exemplos, há uma grande variedade de jogos digitais que podem ser utilizados no ensino de programação. Jogos como o Kodu, Minecraft, Lightbot, entre outros, podem ser usados como ferramentas pedagógicas para o ensino de programação em diferentes níveis e contextos educacionais.

