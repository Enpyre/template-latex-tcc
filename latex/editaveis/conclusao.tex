\chapter[Conclusão]{Conclusão}

Em conclusão, este trabalho apresentou uma pesquisa sobre as dificuldades no ensino de programação e propôs uma solução para mitigar essas dificuldades através da abordagem lúdica com a criação de jogos utilizando a linguagem Python. A ferramenta Enpyre foi desenvolvida com o intuito de oferecer uma opção acessível e prática para professores e alunos que buscam uma forma mais atrativa e eficaz de ensinar e aprender programação. Embora não tenhamos testado a ferramenta, acreditamos que sua criação pode ser uma importante contribuição para a área de ensino de programação e que ela pode ser útil para aqueles que buscam uma abordagem mais dinâmica e envolvente para o aprendizado de programação. Esperamos que este trabalho possa incentivar novas pesquisas e contribuir para o desenvolvimento de ferramentas cada vez mais eficientes e acessíveis para o ensino de programação.


Apesar de fugir do escopo deste trabalho, os próximos passos naturais da Enpyre seriam testar a ferramenta em um ambiente real, de forma a mensurar sua capacidade de ensino de programação.

Por fim, é importante destacar que a aprendizagem de programação é um processo contínuo e que novas metodologias e ferramentas devem ser constantemente avaliadas e atualizadas para garantir que os estudantes tenham acesso ao melhor ensino possível.
