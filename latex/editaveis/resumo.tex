\begin{resumo}
    Este trabalho apresenta uma proposta de ferramenta para minimizar as dificuldades de ensino e aprendizado de programação. A ferramenta é uma plataforma online que permite a execução de código Python diretamente no navegador, eliminando a necessidade de instalação de requisitos técnicos na máquina do aluno. Além disso, a plataforma oferece uma abordagem lúdica para o ensino de programação, através de quiz para fixação de conteúdo e da criação de jogos para exercitar a lógica de programação. A ferramenta foi desenvolvida utilizando a biblioteca Pyodide, que permite a execução de código Python no navegador através da tecnologia Web Assembly. A interface da ferramenta foi desenvolvida utilizando a biblioteca PixiJS, que permite a criação de jogos em 2D no navegador. A ferramenta foi desenvolvida como um Trabalho de Conclusão de Curso e está disponível em \url{https://enpyre-play.vercel.app/}.

 \vspace{\onelineskip}

 \noindent
 \textbf{Palavras-chave}:  Python; Ensino de Programação; Aprendizado de Programação; Web Assembly; PixiJS; Pyodide.
\end{resumo}
