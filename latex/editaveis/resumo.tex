\begin{resumo}
    A programação é uma das disciplinas mais importantes nos cursos de Ciências da Computação e Engenharia de Software, mas também pode ser uma das mais desafiadoras para aprender. Este trabalho destaca as principais dificuldades relacionadas ao ensino e aprendizado de programação e propõe o desenvolvimento de um software que as minimize. A ferramenta proposta permitirá a execução de código Python diretamente no navegador, eliminando a necessidade de instalar requisitos técnicos na máquina do aluno, além de oferecer recursos de desenho em canvas para facilitar a visualização da execução do programa e permitir a interação do aluno.

    Além disso, a plataforma incluirá um quiz para reforçar o aprendizado teórico e projetos práticos que permitirão ao aluno criar jogos iterativos em tempo real e receber feedback sobre seu progresso. O professor também poderá avaliar o aluno com base em suas respostas no quiz e na criação dos projetos.

    A prova de conceito foi realizada com sucesso usando as bibliotecas Pyodide e PixiJS, e os resultados mostraram que a construção da plataforma é viável.

 \vspace{\onelineskip}
    
 \noindent
 \textbf{Palavras-chave}:  Python. Ensino de Programação. Aprendizado de Programação. Web Assembly. PixiJS. Pyodide.
\end{resumo}
