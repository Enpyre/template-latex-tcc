\begin{resumo}
Programação é um dos aspectos mais importantes dos cursos de Ciências da Computação
e Engenharia de Software, ao mesmo tempo em que é um dos temas mais difíceis de se
aprender. Este trabalho destaca algumas das principais dificuldades relacionadas ao ensino e aprendizado de programação, e propõe o desenvolvimento de um software pensado
em minimizar algumas delas. A ferramenta proposta deve permitir a execução de código
Python à partir de um navegador, eliminando-se assim as dificuldades associadas aos requisitos técnicos que devem ser instalados na máquina do aluno. Além disso, deve oferecer
recursos de desenho em canvas, de modo a facilitar a visualização da execução de um programa e permitir a interação do aluno. Foi feita uma prova de conceito, utilizando-se as
bibliotecas Pyodide e PixiJS, e chegou-se à conclusão de que a construção do software é
viável.

 \vspace{\onelineskip}
    
 \noindent
 \textbf{Palavras-chave}:  Python. Ensino de Programação. Aprendizado de Programação. Web Assembly. PixiJS. Pyodide.
\end{resumo}
